% Time to use my expert non-existent rhetoric skills to woo the audience.

\subsection*{Introduction}

The basic structure should be:
\begin{itemize}
    \item Greetings
    \item Explain how the lesson I'm trying to teach isn't quite central to the
        problem itself but rather wide reaching in terms of how to approach math.
    \item Explain the basic genre of the problem and the context (intersection
        of probability and a little bit of physics).
\end{itemize}

What I really aim to teach here isn't so much some new piece of mathematics or
some fancy trick (not that I don't appreciate those now and again), but rather
a way to approach math. Certainly I find the problem I'm presenting interesting
(otherwise I wouldn't be presenting it), but in all honesty whether or not you
like it as much as I do is simply up to you.

Some of this stuff above is probably meh.

The statement of the problem is as such: "If a particle at the origin takes a
path of distance \( 1 \), what is its displacement?" For those not quite
familiar with the specific terminology, a brief way to say it would be that
distance is the numeric value capturing the length of the path, and the
displacement is where the path takes you \textit{(refer to diagram)}. So, in
other words, if a particle takes a path of length 1, where does it end up? I'll
leave you a little time to ponder that.

The answer is that it's a trick question; I think it's rather obvious that you
\textit{can't really know}. In fact, this was question posed to us by Neff (for
those who don't know him, he's the Physics 1 and 2 teacher), and it was a
question posed to demonstrate the differences between distance and
displacement.

That being said, we're mathematicians, so it's not as if we can't get any
information of this. Instead, let us consider the \textit{probability
distribution} of displacements given that the particle chooses some random
path. That is, if all the paths are equally likely to be chosen, what is the
probability that we arrive at any single point?

\subsection*{Properties}

Things to talk about:
\begin{itemize}
    \item How the particle stays inside the circle
    \item Rotational symmetry (problem solving technique)
\end{itemize}

\subsection*{Modeling}

This is where some of the real fun stuff begins. Now that we have an idea in
mind for the problem, we have to really consider: \textit{what is a path?}.
This isn't a trivial question, and I spent a little bit of time formulating and
modeling our paths in a way that could be useful.

The way we're going to do it is break the path into some fixed number of steps
(or line segments) \( N \). By our condition of a length \( 1 \) path, this
means each step has a length of \( 1 / N \). Each step, in turn, is just a
randomly chosen point \( 1 / N \) units away; in other words, we pick uniformly
from all points on a circle of radius of \( 1 / N \).

But obviously, this path is far too coarse for what we want. We want the number
of line segments to go to infinity, with the length of the segments themselves
being infinitesimal, with this all being summed up. In other words, I'm hinting
at this: an integral over a continuum of random variables (we technically call
this a stochastic random process), which honestly looks pretty cool. That being
said, for us to be able to work with this better, we do have to transform this
into the slightly less cool-looking summation.

Now the only problem with evaluating this is that it's really hard. When adding
random variables, you need to consider all infinite cases where points add and
then multiply probabilities. This is bearable in one dimension, but not really
for two dimensions. In fact, with our circle, there is a very geometric view of how
annoying this would be for even just adding two variables.

While that may be a roadblock, it doesn't quite stop us from figuring out the
distribution altogether; we just have to get a bit creative. Here's where we
see another problem solving technique: reducing the dimension or complexity of
the problem in a way that benefits us.

In particular, we can take a single component of the path and use the triangle
inequality to see that it can be anywhere in this radius sized interval. This
turns an awkward circular distribution into a nice uniform distribution. If you
remember, we only need a slice to figure out the whole distribution, so we
aren't losing any information by doing this.

Adding up the \( N \) random variables, we get a simple\textit{-er}
distribution that we can actually graph. And when we do, something very
interesting happens.

As we increase the values of \( N \), the distribution gets pinched further and
further. With this in mind, if we let \( N \) go to infinity the problems true
colors reveal itself.

We get the Dirac delta distribution, defined to have infinite probability at
the origin and zero probability everywhere else. We said earlier that we
couldn't know where the particle was, but in reality it's almost impossible for
it to be anywhere but the origin. Obviously that's not to say it couldn't be
anywhere else, but it is sure darn likely.

Now that I have hopefully wooed you with the amazing and unintuitive result,
I'll show a simulation that probably makes it far more clear why this happens.

\subsection*{Conclusion}

To be a mathematician is to be constantly asking yourself questions. To be a
mathematician is to be constantly learning and out on perhaps an adventure of
knowledge. I've always thought conventional, especially school teaching, of
mathematics to be quite boring and not really representative of math as a whole. While
teaching from the basics does have some of its merits, I've found that creating
and solving my own problems has led me to learn far more than school could
teach me. I mean heck these sorts of probability problems are how I learned
stats and even comp sci a while back.

My wish for you: find a problem that captivates you. It could be small or
ambitious, on a subject that you're familiar with or maybe you're not familiar
with. Maybe it doesn't even have to be related to math. \textit{Find a problem that
captivates you, and enjoy it.} If you can do that, I genuinely do think you'll
learn something amazing.

And with that, there's my Math and Physics presentation. I really enjoyed the
club last year and I hope I can be your officer this year. I'll enjoy the other
presentations; that's all from me.
